\documentclass{article}
\usepackage[utf8]{inputenc}
\usepackage[left=2.5cm,right=2.5cm,top=2.5cm,bottom=2.5cm]{geometry}

\usepackage{notes2bib}
\usepackage{amsmath}
\usepackage{amssymb}
\usepackage{hyperref}
\usepackage{graphicx}

\usepackage{epstopdf}
\usepackage{epsfig}

\usepackage{paralist} % inline list

\usepackage{amsthm} % theorem and definition
\newtheorem{mydef}{Definition}
\newtheorem{exmp}{Example}

\usepackage{tikz}
\usepackage{xytree}

\usepackage{color}
\usepackage{nth}
\usepackage{hhline}
\usepackage{float}
\restylefloat{table}

\title{High-level image interpretation using logical approach and morphological approach}
\author{Yifan YANG}
 
 \begin{document}
 \maketitle
 \section{Introduction}
 Digital image itself is a numerical representation which does not include the semantic information. 
 However, the semantics of an image is the most interesting part for humans to understand the content of the image.
 Moreover, beyond a single object understanding based on object features like colors and forms, etc, we are interested in a complex description which
 relied on context information like spatial relations.
 We need to interpret it with prior knowledge.
 In this thesis, we focuses on diagnostic problem (decision support system) in LOGIMA project
 The knowledge may not be accurate presented and correspond what we are looking at. Lack of expressivity graph representation
 
 \section{Description Logics}
 
 \nocite{*}
 \bibliographystyle{plain}
 \bibliography{mi_ref}
 \end{document}
 